\documentclass[a4paper, 12pt]{article}
\usepackage[left=3cm,right=2cm,bottom=3cm,top=2cm]{geometry}
\usepackage{amsmath}
\usepackage{amssymb}
\usepackage{amsfonts}
\usepackage[dvipsnames]{xcolor}
\usepackage{accents}
\usepackage{esvect}
\usepackage{lastpage}
\usepackage{fancyhdr}
\usepackage{graphicx}
\usepackage[labelfont=bf, justification=centering]{caption}
\usepackage{amsthm}
\usepackage{hyperref}

\newtheorem{thm}{Theorem}[section] % the main one
\newtheorem{lemma}[thm]{Lemma}
% other statement types

% for specifying a name
\theoremstyle{plain} % just in case the style had changed
\newcommand{\thistheoremname}{}
\newtheorem{genericthm}[thm]{\thistheoremname}
\newenvironment{namedthm}[1]
  {\renewcommand{\thistheoremname}{#1}%
   \begin{genericthm}}
  {\end{genericthm}}

%------------ สำหรับการใช้ฟอนต์ภาษาไทย -----------%

\usepackage[no-math]{fontspec}
\usepackage{polyglossia}
\setdefaultlanguage{thai}
\setotherlanguage{english}

\XeTeXlinebreaklocale "th-TH"
\XeTeXlinebreakskip = 0pt plus 0pt
\setmainfont[Scale=MatchLowercase]{TH Sarabun Chula}
\renewcommand{\baselinestretch}{1.3}

\usepackage{tikz}
\usepackage{esint}

\begin{document}

\thispagestyle{plain}
\begin{center}
{\Large\textbf{คู่มือการอ่านบทความ ``Analysis of a Precambrian resonance-stabilized day length''}}\\
อ้างอิง doi: 10.1002/2016GL069812\\
\today
\end{center}
\begin{abstract}
	บทความนี้เป็นบทความที่พูดถึงความหน่วงของการหมุนรอบตัวเองของโลก (Earth's decelerating rotation) ซึ่งได้พูดว่าในช่วงยุค Precambrian ซึ่งเป็นช่วงที่โลกอยู่ในช่วงที่กำลังสร้างสิ่งมีชีวิต (ประมาณก่อน 500 ล้านปีก่อน) จะมีอุณหภูมิที่แปรผันอย่างมากตั้งแต่อุณหภูมิที่สูงและต่ำอย่างมาก ทำให้เป็นปัจจัยที่โลกหมุนรอบตัวเองหน่วงลงอย่างมีนัยสำคัญ ซึ่งแต่เดิมแล้วโลกหมุนรอบตัวเองด้วยอัตราคงที่ที่ประมาณ 21 ชั่วโมง แม้จะมี thermal noise เล็กน้อยแทรกอยู่ตลอดก็ตาม การค้นพบนี้มีการศึกษามาแล้วผ่านการวิเคราะห์ทางธรณีวิทยา เช่น ชั้นหิน, ความเป็นกรดเบส และอุณหภูมิคาดการณ์ที่ทำให้มีลักษณะชั้นหินแบบต่าง ๆ ขึ้นในแต่ละยุคสมัย\footnote{ตามสมมติฐานที่ให้ใน Zahnle and Walker (1987), Crowley (1983)} ความพิเศษของบทความนี้ คือ ผู้วิจัยได้พิจารณาโดยใช้วิธีจำลองออกมาจากสมการคลื่น ซึ่งพิจารณา atmospheric tidal force และ thermal perturbation ด้วยคอมพิวเตอร์ ซึ่งทำให้เห็นภาพแนวโน้มของการหมุนรอบตัวเองในอนาคตได้
\end{abstract}
\tableofcontents
\noindent\rule{\textwidth}{0.4pt}

\section{โครงสร้างของบทความ}
บทความจะแบ่งออกมาได้หยาบ ๆ ได้ทั้งหมด 3 ส่วนด้วยกัน โดยตัดส่วนเกริ่นนำออกไป คือ
\begin{enumerate}
	\item การวิเคราะห์ atmospheric resonance ซึ่งพิจารณาจากกระแสขึ้นลงของบรรยากาศ (atmospheric tide)\footnote{ซึ่งอาจจะไม่เข้าใจว่ามันคืออะไร ให้ลองนึกจินตนาการดูว่ามหาสมุทรจะมีปรากฏการณ์น้ำขึ้นน้ำลง บรรยากาศก็จะมีความสูงนับจากพื้นไปยังอวกาศที่ไม่คงที่เช่นเดียวกันกับน้ำ เพราะน้ำกับอากาศล้วนเป็นของไหลทั้งสิ้น} ซึ่งพิจารณาในหลายปัจจัย แต่หากมีการลดทอนปัจจัยต่าง ๆ ไปบ้างตามสถานการณ์ เช่น ปัญหานี้เป็นปัญหา 1 มิติ มีการสนใจเรื่อง damping coefficient ซึ่งจะทำให้เห็นว่า resonance เกิดขึ้นได้อย่างไรหรือเกิดขึ้นในสถานการณ์ใด
	\item การวิเคราะห์ว่า resonance หายไปเมื่อไหร่ โดยพิจารณาความสัมพันธ์ระหว่างอัตราเร็วเชิงมุมกับการเปลี่ยนแปลงของอุณหภูมิ ซึ่งสามารถคำนวณ torque ที่ดวงจันทร์กระทำต่อโลกได้ ประกบกับ power loss ($Q$-factor) และนำไปจับกับนัยสำคัญทางธรณีวิทยา
	\item การจำลองโมเดลซึ่งเห็นว่าการเปลี่ยนแปลงอุณหภูมิสูง ๆ จะทำให้การหมุนรอบตัวเองเปลี่ยนไป ซึ่งใช้ step size อยู่ที่ 50 ปี (ซึ่งเราจำลองสถานการณ์ในระยะ 1,000 ล้านปี\footnote{20M step sizes}) ซึ่งมีความท้าทายในการเขียนโค้ดที่หาจุดที่เสถียรและไม่เสถียรในการหมุนรอบตัวเองของโลก โดยวิธี multiprocessed binary search ผลที่ได้ออกมานั้นกล่าวได้ว่าจะต้องเกิดขึ้นจากการเปลี่ยนแปลงอุณหภูมิครั้งใหญ่เท่านั้น ซึ่งเหตุการณ์ที่ทำให้เกิดสิ่งนี้ได้ คือ deglaciation เป็นต้น เขาได้เอา 4 snowball events\footnote{โลกกลายเป็นน้ำแข็งทั้งใบ ซึ่งเหตุการณ์ที่ว่าประกอบด้วย Kaigas, Marinoan, Sturtian และ Kirschvik} ในช่วง Precambrian ผลที่ได้มีน่าสนใจเป็นอย่างมาก แต่ยังไม่สามารถพิสูจน์ได้แน่ชัดว่า\footnote{หลักฐานที่บ่งชี้ว่าโลกหมุนรอบตัวเองหน่วงลงใช้หลักฐานปะการังและ Stromatolite ว่าโลกหมุนรอบตัวเองอย่างคงที่ แต่เพิ่มเป็น 24 ชั่วโมงอย่างรวดเร็ว (Williams, 2000) ซึ่ง debunked มาแล้วว่าอาจจะไม่แม่นยำเพียงพอที่จะเอาวิเคราะห์การหมุนรอบตัวเองของโลกแบบเป๊ะ ๆ (Panella, 1972; Zahnle and Walker, 1987; Scruton, 1978; Hofmann, 1983) แต่สามารถวิเคราะห์ในเชิงคุณภาพได้} โลกหมุนด้วยอัตราเร็วเชิงมุมที่เสถียรเมื่อ 2 พันล้านปีก่อน และหลุดจากความเสถียร resonance เมื่อ 600 ล้านปีก่อน (มีเพียงหลักฐานเล็กน้อยที่ Rooney et al., 2014 ได้สรุปมาว่ามี glaciation ที่อาจทำให้เกิด resonance-breaking ได้)
\end{enumerate}

\section{การวิเคราะห์ atmospheric resonance}
ส่วนนี้จะครอบคลุมไปยังหัวข้อที่ 2: Analysis of atmospheric resonance ซึ่งสมการ/แนวคิดทางฟิสิกส์ที่สำคัญที่ต้องรู้ในหัวข้อนี้มีดังต่อไปนี้
\begin{enumerate}
	\item Wave equation ซึ่งพิจารณา Lamb wave: สมการพื้นฐานหน้าตาประมาณนี้,
	$$\frac{1}{c^2}\frac{\partial^2 u}{\partial t^2} = \Delta u$$
	\begin{itemize}
		\item $\Delta := \dfrac{\partial^2}{\partial x_1^2} + \dfrac{\partial^2}{\partial x_2^2} + \cdots + \dfrac{\partial^2}{\partial x_n^2}$ หรือเรียกว่า Laplacian operator ไว้ใช้บอกถึงความโค้งเว้าของรูปร่าง เช่นว่า หาก $\Delta u > 0$ คือ บริเวณนั้นมีความเว้า, $\Delta u < 0$ คือ บริเวณนั้นมีความนูน หรือ $\Delta u = 0$ คือ บริเวณนั้นแบนราบ \textbf{หรือ บริเวณนั้นมีลักษณะเป็นอานม้า}\footnote{เพราะว่าในจุด ๆ หนึ่ง แกนหนึ่งอาจจะเว้า อีกแกนหนึ่งอาจจะนูน ซึ่งทำให้เมื่อพิจารณา Laplacian operator จะถูกหักล้างกันเหลือ 0 ก็ได้}
		\item $\dfrac{\partial^2}{\partial^2 t}$ หมายถึงความเร่งในบริเวณนั้น ๆ (ซึ่งตีความเป็นแรงต่อมวลได้)
		\item $c$ หมายถึงความเร็วของลูกคลื่นที่เราเห็นว่าลูกคลื่นเคลื่อนที่ ซึ่งจะแตกต่างจากความเร่งในบริเวณนั้นที่สนใจเพียงแต่อนุภาคนั้นเท่านั้น (อาจจะเรียกมันว่า group velocity ก็ได้เพื่อป้องกันความสับสน)
		\item หากสนใจที่มาของสมการซึ่งมาจากกฎของนิวตัน สามารถดูต่อได้ที่ \url{https://en.wikipedia.org/wiki/Wave_equation#Derivation_of_the_wave_equation}
	\end{itemize}
	ซึ่งเราสามารถใส่ forced term เข้าไปได้ใน wave equation ด้วยการใส่เทอมความเร่งเพิ่มดังสมการที่ 1 ในบทความ โดยสมการสนใจเพียง 2 มิติ คือ เวลากับแกนปริภูมิเดียวอย่าง $x$ เพราะด้วยว่าเราสามารถละผลกระทบในอีกแกนหนึ่งอย่าง $y$ ได้
	\item Laplace's tidal equation: ไม่สาวต่อ สมการนี้เป็นสมการที่ไว้ใช้อธิบายความสูงและ group velocity ของคลื่นที่อยู่บนตัวกลางที่มีความลึกหนึ่ง ๆ (สามารถศึกษาเพิ่มเติมได้ที่: \url{http://www.ccpo.odu.edu/~klinck/Reprints/PDF/longuetRoySocLonA1968.pdf})
	\item Shallow water equations: สมการนี้ถูกเอามาใช้ใส่แทนค่า $c$ ในสมการคลื่น โดยตัวสมการมาจากสมมติฐานน้ำตื้นและ 1 มิติ ลดรูปจาก Navier-Stokes's equations ซึ่งมีชื่อว่า Saint-Venant equations มีค่าความเร็วคลื่นเท่ากับ $\sqrt{gh}$ (ดูได้ที่ \url{https://en.wikipedia.org/wiki/Shallow\_water\_equations\#Characteristics})
	\item Damped wave oscillation: ให้ดูคอนเซปท์ของสมการ 15.7.1 ถึง 15.7.3 ที่ \url{https://phys.libretexts.org/Bookshelves/University_Physics/Book%3A_University_Physics_(OpenStax)/Book%3A_University_Physics_I_-_Mechanics_Sound_Oscillations_and_Waves_(OpenStax)/15%3A_Oscillations/15.07%3A_Forced_Oscillations} ซึ่งเราเห็นว่าอัตราการหมุนของโลกจะอยู่ตามสมการที่ 2 ของบทความในเทอมของ $\omega$ (ให้สังเกตประโยคว่า ``Near resource, $gh_0 = \frac{4\omega^2}{\beta k^2} \approx \frac{4\omega_0^2}{\beta k^2}$,...'' นี่คือที่มาว่าทำไมช่วงที่อยู่ในสถานะ resonance โลกถึงหมุนรอบตัวเอง 21 ชั่วโมงต่อวัน)
	\item $Q$-factor: คิดว่าอันนี้เข้าใจได้ไม่ยากมาก ขอไม่อธิบายต่อ ให้ไปดูที่ \url{https://en.wikipedia.org/wiki/Q_factor} ซึ่งใช้ตีความ 2 ย่อหน้าสุดท้ายของหัวข้อที่ 2
	\item dissipative Lamb wave forces: ตามสมการที่ 4 ในบทความ เราจะเห็นได้ว่าสมการนี้จะให้ค่า $\Gamma$ (damping coefficient จาก surface interaction ระหว่างแกนโลกกับชั้นแมนเทิล) 
\end{enumerate}

ที่เหลือหน้าตาคล้าย ๆ เดิม แต่เพียงว่าจะต่อยอดแนวคิดนิดหน่อย ไม่ต่างจากเดิมมากนัก

\section{การวิเคราะห์ว่า resonance หายไปเมื่อไหร่}
$\Delta T$ ที่เป็นบวกมาจาก $T_0$ (อุณหภูมิอยู่ในหน่วยเคลวิน) เกิดอะไรขึ้น:
\begin{enumerate}
	\item resonance frequency ลดลง
	\item ความสัมพันธ์การเปลี่ยนแปลงความถี่ resonance เป็นไปตามสมการที่ 7
	\item ตัวแปรที่สามารถตีความได้ คือ $t_w$ เป็นเวลาที่การเปลี่ยนแปลงอุณหภูมิจากเดิมเป็นของใหม่ตามสมการที่ 10 ว่าถ้ามีการเปลี่ยนแปลงด้วยระยะเวลาที่เหมาะสม คาบการหมุนรอบตัวเองของโลกจะโดน resonance breaking
\end{enumerate}
\end{document}